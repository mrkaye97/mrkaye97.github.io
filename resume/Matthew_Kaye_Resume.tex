%%%%%%%%%%%%%%%%%
% This is an sample CV template created using altacv.cls
% (v1.3, 10 May 2020) written by LianTze Lim (liantze@gmail.com). Now compiles with pdfLaTeX, XeLaTeX and LuaLaTeX.
%
%% It may be distributed and/or modified under the
%% conditions of the LaTeX Project Public License, either version 1.3
%% of this license or (at your option) any later version.
%% The latest version of this license is in
%%    http://www.latex-project.org/lppl.txt
%% and version 1.3 or later is part of all distributions of LaTeX
%% version 2003/12/01 or later.
%%%%%%%%%%%%%%%%

%% If you need to pass whatever options to xcolor
\PassOptionsToPackage{dvipsnames}{xcolor}

%% If you are using \orcid or academicons
%% icons, make sure you have the academicons
%% option here, and compile with XeLaTeX
%% or LuaLaTeX.
% \documentclass[10pt,a4paper,academicons]{altacv}

% Use the "normalphoto" option if you want a normal photo instead of cropped to a circle
% \documentclass[10pt,a4paper,normalphoto]{altacv}

\documentclass[10pt,a4paper,ragged2e,withhyper]{/Users/matt/documents/GitHub/mrkaye97.github.io/resume/altacv}
%% AltaCV uses the fontawesome5 and academicons fonts
%% and packages.
%% See http://texdoc.net/pkg/fontawesome5 and http://texdoc.net/pkg/academicons for full list of symbols. You MUST compile with XeLaTeX or LuaLaTeX if you want to use academicons.

% Change the page layout if you need to
\geometry{left=1cm,right=1cm,top=1cm,bottom=1cm,columnsep=1cm}

% The paracol package lets you typeset columns of text in parallel
\usepackage{paracol}

% Change the font if you want to, depending on whether
% you're using pdflatex or xelatex/lualatex
\ifxetexorluatex
% If using xelatex or lualatex:
\setmainfont{Roboto Slab}
\setsansfont{Lato}
\renewcommand{\familydefault}{\sfdefault}
\else
% If using pdflatex:
\usepackage[rm]{roboto}
\usepackage[defaultsans]{lato}
% \usepackage{sourcesanspro}
\renewcommand{\familydefault}{\sfdefault}
\fi



% Change the colours if you want to
\definecolor{SlateGrey}{HTML}{2E2E2E}
\definecolor{LightGrey}{HTML}{666666}
\definecolor{DarkPastelRed}{HTML}{989898}
\definecolor{PastelRed}{HTML}{FF4D4D}
\definecolor{GoldenEarth}{HTML}{A9A9A9}
\colorlet{name}{black}
\colorlet{tagline}{PastelRed}
\colorlet{heading}{DarkPastelRed}
\colorlet{headingrule}{GoldenEarth}
\colorlet{subheading}{PastelRed}
\colorlet{accent}{PastelRed}
\colorlet{emphasis}{SlateGrey}
\colorlet{body}{LightGrey}

% Change some fonts, if necessary
\renewcommand{\namefont}{\Huge\rmfamily\bfseries}
\renewcommand{\personalinfofont}{\footnotesize}
\renewcommand{\cvsectionfont}{\LARGE\rmfamily\bfseries}
\renewcommand{\cvsubsectionfont}{\large\bfseries}


% Change the bullets for itemize and rating marker
% for \cvskill if you want to
\renewcommand{\itemmarker}{{\small\textbullet}}
\renewcommand{\ratingmarker}{\faCircle}


\begin{document}
	\name{Matt Kaye}
	
	%% You can add multiple photos on the left or right
	% \photoL{2.5cm}{Yacht_High,Suitcase_High}
	
	\personalinfo{%
		% Not all of these are required!
		\homepage{mrkaye97.github.io}
		\github{mrkaye97}
		\email{mrkaye97@gmail.com}
		\phone{(646) 853-5997}
		\linkedin{kayem20}
		%% You MUST add the academicons option to \documentclass, then compile with LuaLaTeX or XeLaTeX, if you want to use \orcid or other academicons commands.
		% \orcid{0000-0000-0000-0000}
		%% You can add your own arbtrary detail with
		%% \printinfo{symbol}{detail}[optional hyperlink prefix]
		% \printinfo{\faPaw}{Hey ho!}[https://example.com/]
		%% Or you can declare your own field with
		%% \NewInfoFiled{fieldname}{symbol}[optional hyperlink prefix] and use it:
		% \NewInfoField{gitlab}{\faGitlab}[https://gitlab.com/]
		% \gitlab{your_id}
	}


	\makecvheader
	%% Depending on your tastes, you may want to make fonts of itemize environments slightly smaller
	% \AtBeginEnvironment{itemize}{\small}
	
	%% Set the left/right column width ratio to 6:4.
	\columnratio{0.7}
	
	% Start a 2-column paracol. Both the left and right columns will automatically
	% break across pages if things get too long.
	\begin{paracol}{2}
		\cvsection{Experience}
	
		\cvevent{Data Scientist}{CollegeVine}{Sept 2020 -- Present}{}
		\begin{itemize}
		\item First member of the CollegeVine data science team. Responsible for establishing data science best practices, building out modeling pipelines, helping teammates do rigorous data analysis, and explaining data scientific methods to other members of the team to democratize data scientific work at CollegeVine
		\item Transitioned our chancing algorithm (used by hundreds of thousands of students to get their chances at over 1,500 colleges and universities) away from a rules- and heuristics-based model to an ML-forward approach. Built expert domain knowledge into the model features and made product-first modeling choices to ensure intuitive model behavior for users
		\item Led a sequential testing revolution and built out a dashboard with sequential testing tools to help our PMs run A/B tests more quickly without sacrificing statistical rigor
		\item Responsible for owning all components of the data science process, including wrangling data, doing data analysis, building and validating models, deploying models to production (generally as Dockerized microservices on Heroku), and monitoring models and APIs in production
		\item Serving as a subject matter expert for everyday statistics- and data science-related questions and problems
		\end{itemize}
		
		\divider
		
		\cvevent{Open-Source R Developer}{slackr \& fitbitr}{Oct 2020 -- Present}{}
		
		\begin{itemize}
		\item Current author and maintainer of $slackr$, an R package for connecting R to Slack with 215k+ downloads
		\item Author and creator of $fitbitr$, an R package that streamlines pulling Fitbit user data via the Fitbit API
		\item Responsible for all aspects of package development and maintenance, including implementing new methods, improving error handling and messaging, writing unit tests, establishing and maintaining a CI/CD pipeline, writing descriptive documentation, helping users work through issues and bugs, reviewing PRs, and more
		\end{itemize}
		
		\divider

		\cvevent{Baseball Operations Fellow}{Baltimore Orioles}{Mar 2020 -- Sept 2020}{}
		\begin{itemize}
	%		\item Solving modeling problems with a number of statistical and machine learning techniques, including Bayesian gamma hurdle regression, XGBoost and random forest classifiers, and nearest neighbor algorithms
	%		\item Applying industry knowledge to improve model performance
		\item Created a fully Bayesian, simulation-based projection system for MLB player performance over a six year time horizon
		\item Modeled free agent salaries with a gamma hurdle regression framework
		\item Contributed to a Markov Chain Monte Carlo approach to determining optimal shifts against opposing hitters
		\item Worked on a variety of day-to-day data science tasks related to game strategy and player evaluation
		\end{itemize}

		
%		\cvsection{Projects}
%		
%		\cvevent{Project 1}{Funding agency/institution}{}{}
%		\begin{itemize}
%			\item Details
%		\end{itemize}
%		
%		\divider
%		
%		\cvevent{Project 2}{Funding agency/institution}{Project duration}{}
%		A short abstract would also work.
%		
%		\medskip
		
		\switchcolumn
		\smallskip
		\cvsection{Skills}
		\textbf{Programming Languages:}\\
		\bigskip

		\cvskill{R}{5}
		\divider
		
		\cvskill{Python}{4}
		\divider
		
		\cvskill{SQL}{4}
		\divider
		
		\cvskill{Shell Scripting}{3}
		\divider
		
		\cvskill{Haskell}{2}
		\divider
		
		\cvskill{Java}{2}\\
		
		
		\divider\smallskip\\
		\textbf{Frameworks, Software, and Tools:}\\
		\smallskip\\
		\smallskip\\
		
		\cvtag{AWS (S3, Redshift, RDS, etc.)}\\
		\cvtag{brms}
		\cvtag{Docker}
		\cvtag{Heroku}
		\cvtag{Numpy} \\
		\cvtag{Pandas}
		\cvtag{R Shiny}
		\cvtag{Scikit-Learn} \\
		\cvtag{Tidyverse + Tidymodels}\\
		
		
		\divider\smallskip\\
		\textbf{Data Science:}\\
		\smallskip
		\smallskip
		
		\cvtag{Bayesian Modeling and Inference}\\
		\cvtag{Data Wrangling}
		\cvtag{Data Visualization}\\
		\cvtag{Machine Learning}\\
		\cvtag{Time Series Modeling}\\
		
		%% Yeah I didn't spend too much time making all the
		%% spacing consistent... sorry. Use \smallskip, \medskip,
		%% \bigskip, \vpsace etc to make ajustments.
		\medskip
		
		\cvsection{Education}
		
		\cvevent{Bachelor of Arts\\ Economics and Mathematics}{Carleton College}{Sept 2016 -- Nov 2019}{}

		\divider
		
		\cvevent{}{Choate Rosemary Hall}{Sept 2013 -- June 2016}{}
		
		
		\cvsection{Interests}
		
		\cvtag{Baseball}
		\cvtag{Browsing r/AskReddit}\\
		\cvtag{Cooking}
		\cvtag{Running}
		\cvtag{Minimalism}\\
		\cvtag{Skiing}
		\cvtag{Data Visualization}\\
		\cvtag{Fantasy Novels}
		\cvtag{Solo Travel}\\
		\cvtag{Nature \& Architecture Photography}\\
		
		
	\end{paracol}
	
	
\end{document}
