%%%%%%%%%%%%%%%%%
% This is an sample CV template created using altacv.cls
% (v1.3, 10 May 2020) written by LianTze Lim (liantze@gmail.com). Now compiles with pdfLaTeX, XeLaTeX and LuaLaTeX.
%
%% It may be distributed and/or modified under the
%% conditions of the LaTeX Project Public License, either version 1.3
%% of this license or (at your option) any later version.
%% The latest version of this license is in
%%    http://www.latex-project.org/lppl.txt
%% and version 1.3 or later is part of all distributions of LaTeX
%% version 2003/12/01 or later.
%%%%%%%%%%%%%%%%

%% If you need to pass whatever options to xcolor
\PassOptionsToPackage{dvipsnames}{xcolor}

%% If you are using \orcid or academicons
%% icons, make sure you have the academicons
%% option here, and compile with XeLaTeX
%% or LuaLaTeX.
% \documentclass[10pt,a4paper,academicons]{altacv}

% Use the "normalphoto" option if you want a normal photo instead of cropped to a circle
% \documentclass[10pt,a4paper,normalphoto]{altacv}

\documentclass[10pt,a4paper,ragged2e,withhyper]{/Users/matt/Google Drive/GitHub/mrkaye97.github.io/resume/altacv}
%% AltaCV uses the fontawesome5 and academicons fonts
%% and packages.
%% See http://texdoc.net/pkg/fontawesome5 and http://texdoc.net/pkg/academicons for full list of symbols. You MUST compile with XeLaTeX or LuaLaTeX if you want to use academicons.

% Change the page layout if you need to
\geometry{left=1.25cm,right=1.25cm,top=1.5cm,bottom=1.5cm,columnsep=1.2cm}

% The paracol package lets you typeset columns of text in parallel
\usepackage{paracol}

% Change the font if you want to, depending on whether
% you're using pdflatex or xelatex/lualatex
\ifxetexorluatex
% If using xelatex or lualatex:
\setmainfont{Roboto Slab}
\setsansfont{Lato}
\renewcommand{\familydefault}{\sfdefault}
\else
% If using pdflatex:
\usepackage[rm]{roboto}
\usepackage[defaultsans]{lato}
% \usepackage{sourcesanspro}
\renewcommand{\familydefault}{\sfdefault}
\fi



% Change the colours if you want to
\definecolor{SlateGrey}{HTML}{2E2E2E}
\definecolor{LightGrey}{HTML}{666666}
\definecolor{DarkPastelRed}{HTML}{989898}
\definecolor{PastelRed}{HTML}{FF4D4D}
\definecolor{GoldenEarth}{HTML}{A9A9A9}
\colorlet{name}{black}
\colorlet{tagline}{PastelRed}
\colorlet{heading}{DarkPastelRed}
\colorlet{headingrule}{GoldenEarth}
\colorlet{subheading}{PastelRed}
\colorlet{accent}{PastelRed}
\colorlet{emphasis}{SlateGrey}
\colorlet{body}{LightGrey}

% Change some fonts, if necessary
\renewcommand{\namefont}{\Huge\rmfamily\bfseries}
\renewcommand{\personalinfofont}{\footnotesize}
\renewcommand{\cvsectionfont}{\LARGE\rmfamily\bfseries}
\renewcommand{\cvsubsectionfont}{\large\bfseries}


% Change the bullets for itemize and rating marker
% for \cvskill if you want to
\renewcommand{\itemmarker}{{\small\textbullet}}
\renewcommand{\ratingmarker}{\faCircle}


\begin{document}
	\name{Matthew Kaye}
	
	%% You can add multiple photos on the left or right
	% \photoL{2.5cm}{Yacht_High,Suitcase_High}
	
	\personalinfo{%
		% Not all of these are required!
		\homepage{mrkaye97.github.io}
		\email{mrkaye97@gmail.com}
		\phone{(646) 853-5997}
		\linkedin{kayem20}
		\github{mrkaye97}
		%% You MUST add the academicons option to \documentclass, then compile with LuaLaTeX or XeLaTeX, if you want to use \orcid or other academicons commands.
		% \orcid{0000-0000-0000-0000}
		%% You can add your own arbtrary detail with
		%% \printinfo{symbol}{detail}[optional hyperlink prefix]
		% \printinfo{\faPaw}{Hey ho!}[https://example.com/]
		%% Or you can declare your own field with
		%% \NewInfoFiled{fieldname}{symbol}[optional hyperlink prefix] and use it:
		% \NewInfoField{gitlab}{\faGitlab}[https://gitlab.com/]
		% \gitlab{your_id}
	}


	\makecvheader
	%% Depending on your tastes, you may want to make fonts of itemize environments slightly smaller
	% \AtBeginEnvironment{itemize}{\small}
	
	%% Set the left/right column width ratio to 6:4.
	\columnratio{0.6}
	
	% Start a 2-column paracol. Both the left and right columns will automatically
	% break across pages if things get too long.
	\begin{paracol}{2}
		\cvsection{Experience}
	
		\cvevent{Data Scientist}{CollegeVine}{Sept 2020 -- Present}{Boston, MA}
		\begin{itemize}
		\item First member of the CollegeVine data science team
		\end{itemize}
		
		\divider
		
		\cvevent{Baseball Operations Fellow}{Baltimore Orioles}{Mar 2020 -- Sept 2020}{Baltimore, MD}
		\begin{itemize}
	%		\item Solving modeling problems with a number of statistical and machine learning techniques, including Bayesian gamma hurdle regression, XGBoost and random forest classifiers, and nearest neighbor algorithms
	%		\item Applying industry knowledge to improve model performance
		\item Created a fully Bayesian, simulation-based projection system for MLB player performance over a six year time horizon
		\item Created an Gradient Boosting- and simulation-based projection system for a minor league player's MLB performance
		\item Created a gamma hurdle regression framework to predict free agent salaries
		\item Implemented a robust-to-multimodality version of the Metropolis-Hastings algorithm to determine the optimal way to position our defense against a specific hitter
		\item Worked on a variety of day-to-day data science tasks related to game strategy and player evaluation
		\end{itemize}
		
		\divider
		
		\cvevent{Research Intern}{Federal Reserve Bank of St. Louis}{June 2019 -- Aug 2019}{St. Louis, MO}
		\begin{itemize}
			\item Prepared new data series from the BLS, IMF, and more for upload to FRED, using pandas for data wrangling
			\item Fit neural network and multinomial naïve Bayes models to apply correct descriptive tags to FRED series to improve search performance
		\end{itemize}
		
		\divider
		
		\cvevent{Econometrics Teaching Assistant}{Carleton College Department of Economics}{Mar 2019 -- June 2019}{Northfield, MN}
		\begin{itemize}
			\item Ran semiweekly sessions to help econometrics students master course material, improve their R and \LaTeX \space skills, and hone their statistical intuition.
			\item Helped students complete homework assignments and projects and prepare for exams
			\item Graded homework assignments
		\end{itemize}
		
		\divider
		
		\cvevent{Statistics Lab Assistant}{Carleton College Department of Mathematics}{Sept 2018 -- June 2019}{Northfield, MN}
		\begin{itemize}
		\item Explained and walked through examples of concepts in statistics courses, including (but not limited to) data cleaning, hypothesis testing, exploratory data analysis and visualization, and modeling
		\item Worked with students on important R and debugging skills
		\end{itemize}

		
%		\cvsection{Projects}
%		
%		\cvevent{Project 1}{Funding agency/institution}{}{}
%		\begin{itemize}
%			\item Details
%		\end{itemize}
%		
%		\divider
%		
%		\cvevent{Project 2}{Funding agency/institution}{Project duration}{}
%		A short abstract would also work.
%		
%		\medskip
		
		\switchcolumn
		
		\cvsection{Skills}
		
		\cvtag{R}
		\cvtag{Python}
		\cvtag{Git}
		\cvtag{SQL}\\
		\cvtag{Java}
		\cvtag{Excel}
		\cvtag{Tidyverse}\\
		\cvtag{Pandas / Numpy / Scipy}
		
		\divider\smallskip
		
		\cvtag{Data Wrangling}
		\cvtag{Data Visualization}\\
		\cvtag{Bayesian Modeling and Inference}\\
		\cvtag{Machine Learning}
		
		\cvsection{Languages}
		
		\cvskill{English}{5}
		\divider
		
		\cvskill{Spanish}{4}
		\divider
		
		\cvskill{Swedish}{2}
		\divider
		
		\cvskill{French}{1}
		\divider
		
		\cvskill{Portuguese}{1}
		
		%% Yeah I didn't spend too much time making all the
		%% spacing consistent... sorry. Use \smallskip, \medskip,
		%% \bigskip, \vpsace etc to make ajustments.
		\medskip
		
		\cvsection{Education}
		
		\cvevent{B.A.\ Economics and Mathematics}{Carleton College}{Sept 2016 -- Nov 2019}{}
		Economics Thesis: \textit{The Effect of Transit on Life Satisfaction: A Multilevel Modeling Exploration of Urban Happiness}\\
		\smallskip
		Mathematics Capstone: \textit{Dynamic Linear Models and the Kalman Filter}
		
		\divider
		
		\cvevent{}{Choate Rosemary Hall}{Sept 2013 -- June 2016}{}
		
		
		\cvsection{Interests}
		
		\cvtag{Baseball}
		\cvtag{Browsing r/AskReddit}\\
		\cvtag{Cooking}
		\cvtag{Running}
		\cvtag{Minimalism}\\
		\cvtag{Skiing}
		\cvtag{Data Visualization}\\
		\cvtag{Fantasy Novels}
		\cvtag{Solo Travel}\\
		\cvtag{Nature \& Architecture Photography}\\
		
		
	\end{paracol}
	
	
\end{document}
